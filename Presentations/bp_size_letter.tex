\documentclass[preprint2]{aastex}
\usepackage{natbib}
\usepackage{xcolor}
\usepackage{booktabs}
\bibliographystyle{apj}

\pagenumbering{gobble}

\shorttitle{Bright Point Size}
\shortauthors{Farris}

\begin{document}

\title{Determining coronal bright point size via cross-correlation using
multi-wavelength images from AIA/\textit{SDO}}
\author{Laurel Farris, R. T. James McAteer}
\affil{New Mexico State University}
\email{laurel07@nmsu.edu}

\begin{abstract}
\end{abstract}
\keywords{Sun: corona{-}Sun: bright points{-}Sun: multi-wavelength}



\section{Introduction}\label{intro}

% Observed

Coronal bright points (BPs)
are seen ubiquitously in the solar corona in the X-ray and EUV
wavelength regimes, with a more homogeneous and numerous distribution during
solar minimum (\cite{Priest}).

Though they only cover about 1.6 \% of the
photosphere (\cite{Srivastava}), BPs (together with sunspots)
contribute over 90\% of the total magnetic flux (\cite{Howard}).
Over the course of the solar cycle, they can contribute significantly to the
global intensity variation of the sun, particularly in the ultraviolet
regime (\cite{Riethmuller}).

% What they are

They are thought to be composed of bundles of coronal loops, and can be
distinguished into two components: a bright center and a surrounding darker
region (\cite{Zhang}; \cite{Alipour}).

Flashes of emission, or ``jets'' have been observed around these BPs, with
characteristic periods of about one hour (\cite{Zhang}).


% Other methods to determine size in literature
Several techniques for determining the size of coronal BPs have been investigated
in the literature.
\cite{Alipour} developed an algorithm to locate BPs in the corona, using size determined
by intensity as part of the criteria for distinguishing BPs from other features,
such as top-down views of coronal loops or nanoflares.


% This project, outline
Here, the size of a BP was determined using
the cross-correlation between the pixels in and around the visible
area occupied by the BP in the first image of each wavelength.
The cross-correlation was calculated over time for a total of one hour.
The data utilized here is described in \S{} \ref{data},
the results are discussed in \S{} \ref{results},
the analysis in \S{} \ref{analysis},
and the primary conclusions are wrapped up in \S{} \ref{conclusion}.


\section{Data}\label{data}
\begin{table}[h]
\centering
    \begin{tabular}{l l l}
        \hline\hline
        $\lambda$ [\AA{}] & log(T) [K] & Ion \\
        \hline
        94 & 6.8 & Fe {\small XVIII} \\
        131 & 5.6, 7.0 & Fe {\small VIII, XXI} \\
        171 & 5.8 & Fe {\small IX} \\
        193 & 6.2, 7.3 & Fe {\small XII, XXIV} \\
        211 & 6.3 & Fe {\small XIV} \\
        304 & 4.7 & He {\small II} \\
        335 & 6.4 & Fe {\small XVI} \\
    \end{tabular}
\caption{Characteristic temperatures corresponding to the wavelengths observed
    in emission in the solar corona (from \cite{Lemen}).}
\label{temps}
\end{table}

This analysis was carried out using multi-wavelength data from AIA/\textit{SDO}
spanning one hour on June 1, 2012 from 13:00:00 to 13:59:59, at a cadence of 12
seconds.
The downloaded data was processed at level 1.0.
A grayscale image of the full disk at the beginning of the time series for each
bandpass is shown in figure \ref{full}.
A single BP was selected from the coronal hole in the upper
left region of the solar disk.
100 pixel$^{2}$ ($\sim$ 60 arcsecond$^{2}$)
images of this BP in each passband are shown in figure \ref{bp_images}.
The relevant values for each passband are given in table \ref{temps}.

Before analysis,
an alignment procedure was run on each data cube to correct for pixel shifts between
images due to the rotation of the sun.


\begin{figure*}[htb!]
    \includegraphics[width=\textwidth]{Figures/full.png}
    \caption{Full disk showing the relative intensity at the start of the time
        series for each AIA bandpass used in this study. These images show the
        square root of the exact data values for better visualization of the features. }
    \label{full}
\end{figure*}

\begin{figure*}[htb!]
    \includegraphics[width=\textwidth]{Figures/images.png}
    \caption{Images of the BP in six different AIA wavelengths. The pixel coordinates
        are given relative to the full disk shown in figure \ref{full}. The
        location of the BP appears to shift from one bandpass to next, which
        may indicate a structure that is not completely straight, or a possible
        shift in the data itself. As with figure \ref{full}, these images show
        the square root of the original data values.}
    \label{bp_images}
\end{figure*}



\section{Analysis}\label{analysis}
\subsection{Cross-correlation}\label{cc_analysis}
The cross-correlation was calculated over time between the light curves of two
individual pixels at a time. The central pixel of the BP was chosen arbitrarily based on
the intensity of the initial image in each time series. The cross-correlation procedure was
run over the lightcurve of this pixel and every other pixel in the 100 pixel$^{2}$ data
cube, and then repeated for
the four pixels immediately adjacent to the central one.

The purpose of the cross-correlation analysis was to help
determine, at the lowest possible resolution, which parts of the BP were moving
together as a single physical structure. The timelag at which the highest correlation
occurs was expected to be of the same order as the acoustic crossing time.
Since
magnetism dominates the behavior of waves produced in the solar corona, the
speed at which a wave moves through a feature there is expected
to be close to that of the characteristic Alfv\'en speed.



The intensity of each BP compared to the background flux surrounding it
can give a visual estimate
of its size. The intensity of the first image in each
passband is plotted as a function of radial distance from the central pixel in figures
\ref{intensity_1} and \ref{intensity_2}.





\section{Results}\label{results}

Images illustrating the highest cross-correlation value of each pixel and the
timelag corresponding to that correlation value are shown in figures \ref{cc_all}
and \ref{tt_all}, respectively. The correlation was cut off at 0.5 and rescaled to obtain a
better illustration of the structure of the BP. These images are shown in
figures \ref{cc} and \ref{tt}.

As \cite{Alipour} noted, the BP structure is most evident for the
131\AA{}, 193\AA{}, and 211\AA{} images.

\subsection{Variation in BP size with temperature}
Even though the temperature of the corona increases with height from the transition
region to wherever the corona actually ends (need source here), there is not necessarily
a direct correlation between temperature and absolute height above the photosphere
due to the variety of structures
and overall inhomogeneity that exists in the solar atmosphere (\textcolor{red}{source}).
However, the \emph{relative} height between each bandpass for a given structure is
possible to determine, and is relevant to this study.

\subsection{Extra emission in 211\AA{}}
The 211\AA{} data is particularly notable as it shows strong correlation values
at two points to the right of where the BP appears visually in figure \ref{full}.
This could be the result of a jet of light emitted from the main structure, or a
possible indicator of the main structure splitting into several tubes at the
height where the 211 \AA{} emission is strongest.
A movie showing all images from this wavelength (\cite{ssw}) revealed a flash
of emission around the 45th image,
which matches the timelag at which the high correlation values occurred as shown
in figure \ref{closeup}.

\subsection{Crossing time}
A test on this method can be done by using the derived size and the timelag across
the BP to calculate a velocity. In the corona, this should be close to the
Alfv\'en velocity, around 1000 km s$^{-1}$. (With a cadence of 12 seconds, do
we have enough sampling to do this?)

\begin{figure*}[htb!]
    \includegraphics[width=\textwidth]{Figures/BandWimages.png}
    \caption{Images showing the highest cross-correlation value for each pixel. }
    \label{cc_all}
\end{figure*}

\begin{figure*}[htb!]
    \includegraphics[width=\textwidth]{Figures/cc_all.png}
    \caption{Color images showing the highest cross-correlation value for each pixel. }
    \label{cc_all_color}
\end{figure*}

\begin{figure*}[htb!]
    \includegraphics[width=\textwidth]{Figures/tt_all.png}
    \caption{Images showing the timelag corresponding to the correlation values
        illustrated in figure \ref{cc_all}.}
    \label{tt_all}
\end{figure*}

\begin{figure*}[htb!]
    \includegraphics[width=\textwidth]{Figures/cc_images.png}
    \caption{Cross-correlation images scaled to show only values higher than 0.5.}
    \label{cc}
\end{figure*}

\begin{figure*}[htb!]
    \includegraphics[width=\textwidth]{Figures/tt_images.png}
    \caption{Timelag corresponding to the cross-correlation values higher than 0.5.}
    \label{tt}
\end{figure*}

\begin{figure*}[htb!]
    \includegraphics[width=\textwidth]{Figures/im_211.png}
    \caption{Images from the 211 \AA{} bandpass only, around the times when the
        two jets of light appeared in the upper right region of the main body of the BP. }
    \label{211_images}
\end{figure*}

\begin{figure*}[htb!]
    \includegraphics[width=\textwidth]{Figures/closeup_211.png}
    \caption{Timelag at 211 \AA{} ``zoomed in'' around the two jets of light.}
    \label{closeup}
\end{figure*}

\begin{figure*}[htb!]
    \includegraphics[width=\textwidth]{Figures/intensity_1.png}
    \caption{Intensity of each pixel is plotted as a function of radius for each
        passband. I have no idea what's going on with the 94\AA{} data.}
    \label{intensity_1}
\end{figure*}

\begin{figure*}[htb!]
    \includegraphics[width=\textwidth]{Figures/intensity_2.png}
    \caption{Same as figure \ref{intensity_1}, but with half the radius range
        cut off to better view the values around the main BP.}
    \label{intensity_2}
\end{figure*}

\begin{figure*}[htb!]
    \includegraphics[width=\textwidth]{Figures/cc_tt_plot.png}
    \caption{The highest cross-correlation value of each pixel is plotted as a function
        of its distance from the center pixel. The color indicates the timelag
        corresponding to the maximum cross-correlation for that pixel.}
    \label{tt_all_plot}
\end{figure*}

\begin{figure*}[htb!]
    \includegraphics[width=\textwidth]{Figures/cc_tt_plot_scaled.png}
    \caption{Same as figure \ref{tt_all_plot}, but with two thirds of the timelag cut out
        at both ends.}
    \label{tt_plot}
\end{figure*}


\section{Conclusion}\label{conclusion}

\bibliography{reffile}
\end{document}
